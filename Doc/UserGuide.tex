\documentclass[a4paper,12pt]{extarticle}
\usepackage[a4paper]{geometry}
\geometry{verbose,tmargin=2cm,bmargin=2cm,lmargin=2cm,rmargin=2cm}

\usepackage{fontspec}
\defaultfontfeatures{Ligatures=TeX}
%\setmainfont{Linux Libertine O}
\setmainfont{FreeSerif}
%\setmonofont{Fira Mono}
\setmonofont{FreeMono}

\setlength{\parindent}{0cm}

\usepackage{hyperref}
\usepackage{url}
\usepackage{xcolor}

\usepackage{minted}
%\newminted{julia}{breaklines,fontsize=\footnotesize}
\newminted{julia}{breaklines}

\begin{document}

\title{User Guide for {\ttfamily ffr-ElectronicStructure.jl}}
\author{Fadjar Fathurrahman}
\date{}
\maketitle

\tableofcontents

\section{Introduction}

{\ttfamily ffr-ElectronicStructure.jl} is a collection of programs\footnote{or scripts}
for learning electronic structure calculations.


\section{Short introduction to {\tt Julia}}

ℍ

\begin{juliacode}
ℍ
type CircleT
  radius::Float64
end
\end{juliacode}

\begin{Verbatim}[commandchars=\\\{\}]
\PYG{n}{ℍ}
\PYG{k}{type}\PYG{n+nc}{ CircleT}
  \PYG{n}{radius}\PYG{p}{::}\PYG{k+kt}{Float64}
\PYG{k}{end}
\end{Verbatim}


\section{Formulae}

Plane wave basis $b_{\alpha}(\mathbf{r})$:
\begin{equation}
b_{\alpha}(\mathbf{r}) = \frac{1}{\sqrt{\Omega}} e^{\mathbf{G}_{\alpha}\cdot\mathbf{r}}
\end{equation}

Structure factor:
\begin{equation}
S_{I}(\mathbf{G}) = \sum_{\mathbf{G}} e^{ -\mathbf{G}\cdot\mathbf{X}_{I} }
\end{equation}



\end{document}
